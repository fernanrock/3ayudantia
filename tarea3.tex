\documentclass[10pt]{article} 

\usepackage[utf8]{inputenc}

\usepackage{geometry}
\geometry{letterpaper} 

\usepackage[spanish]{babel} % Permite utilizar español
\usepackage{graphicx} % support the \includegraphics command and options
\newcommand{\HRule}{\rule{\linewidth}{0.5mm}}
\usepackage{wrapfig}
\usepackage{float}
\usepackage{verbatim} % adds environment for commenting out blocks of text & for better verbatim

%\usepackage{booktabs}
%\usepackage{array}
\usepackage{subfig}

\usepackage{fancyhdr} % This should be set AFTER setting up the page geometry
\pagestyle{fancy} % options: empty , plain , fancy
\renewcommand{\headrulewidth}{0pt} % customise the layout...
\lhead{}\chead{}\rhead{}
\lfoot{}\cfoot{\thepage}\rfoot{}


\usepackage{sectsty}
\allsectionsfont{\sffamily\mdseries\upshape} 

\usepackage[nottoc,notlof,notlot]{tocbibind} % Put the bibliography in the ToC
\usepackage[titles,subfigure]{tocloft} % Altera el estilo del  indice
\usepackage{amsmath}


\begin{document}


\begin{titlepage}
\begin{center}
\includegraphics[width=0.15\textwidth]{Utem}~\\[1cm]

\textsc{\normalsize Universidad Tecnológica Metropolitana}\\[1.5cm]

\textsc{\ Trabajo 3 Ayudantia }\\[0.5cm]

% Title
\HRule \\[0.4cm]
{ \huge \bf Patrones de Diseño}\\[0.4cm]

\HRule \\[1.5cm]
\textsc{\large Fernando Guerrero Muñoz , David Muñoz Muñoz, Ivan Montenegro }\\[1cm]
\textsc{\large Profesor: Luis Herrera}\\
\textsc{\large Ayudante: Guillermo Rojas}\\
\end{center}
\end{titlepage}

%\tableofcontents Esto Crea el Indice
\newpage

\begin{titlepage}
{ \huge  Relacion de principales patrones GOF (GANG OF FOUR):}

\section{Patrones Creacionales:}
\begin{itemize}


\item Object Pool. 
\item Abstract Factory (fàbrica abstracta).
\item Builder (constructor virtual).
\item Factory Method (Metodo de fabricación).
\item Prototipo.
\item Instancia única (Singleton).


\end{itemize}
\end{titlepage}


\newpage
\section{Patrones Estructurales:}
\begin{itemize}


\item Adaptador o envoltorio (Adapter): oficia de intermediario entre dos clases cuyas interfaces 
son incompatibles de manera tal que puedan ser utilizadas en 
conjunto. 
\item Puente (Bridge): Disocia un componente complejo en dos jerarquías de 
clases: una abstracción funcional y la implementación interna, para 
que ambas puedan variar independientemente.
\item Objeto Compuesto (Composite): Compone objetos en estructuras de árboles para 
representar jerarquías parte-todo. Permite que los clientes traten de 
manera uniforme a los objetos individuales y a los complejos. 
\item Decorador (Decorator): Añade funcionalidad a una clase dinámicamente.
\item Fachada (Facade): Proporciona una interfaz simplificada para un conjunto de 
interfaces de subsistemas. Define una interfaz de alto nivel que 
hace que un subsistema sea más fácil de usar
\item Peso Ligero (Flyweigth): Permite el uso de un gran número de objetos de grano 
fino de forma eficiente mediante compartimiento.
\item Proxy: Provee un sustituto o representante de un objeto para 
controlar el acceso a éste. Este patrón posee las siguientes variantes:
\begin{itemize} 
\item Proxy remoto: se encarga de representar un objeto remoto 
como si estuviese localmente. 
\item Proxy virtual: se encarga de crear objetos de gran tamaño bajo 
demanda. 
\item Proxy de protección: se encarga de controlar el acceso al objeto 
representado.
\end{itemize}

\item Modulo.

\end{itemize}






\newpage
\section{Ejemplo}


\begin{itemize}


\item Si bien es cierto que se pueden colgar pinturas, cuadros y

fotos en las paredes sin marcos, éstos suelen ser utilizados a

menudo y son ellos los que se cuelgan en la pared en lugar de su

contenido (pinturas, cuadros, etc.). Al momento de colgarse los

cuadros junto con su marco pueden formar un solo “componente

visual”(Patron Estructural Decorator)
\end{itemize}



\




\end{document}
